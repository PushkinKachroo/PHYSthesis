\documentclass[oneside,12pt]{book}
    % The option oneside is required for UNLV theses.
    % The 12pt is optional -- UNLV will allow 10pt too.
    % 10pt is the default, so to switch to 10pt, just delete 12pt in the above.
\usepackage{amsthm,amssymb,amsmath,graphicx} % Optional packages.
\usepackage{setspace,UNLVthesis} % Required packages.  Look at UNLVthesis.sty to see how
    % LaTeX is instructed to set things up.  This file may need some tweaking.
    % setspace.sty is not normally part of MikTeX.  It can be obtained from www.ctan.org.
    % Do a search on setspace.sty.
    % The files setspace.sty and UNLVthesis.sty should be in the same directory
    % as this file (or in a directory of MikTeX where LaTeX will know to find it
    % -- for example, where other style files are).
\pagestyle{unlv} %  This is defined in UNLVthesis.sty.  Headings are empty except for page numbers,
    % and the page numbers are the same size as the text.  Most documents use a different size so that
    % it it is difficult to mistake it as part of the text.

%These define the format and numbering of theorem like environments.
\newtheorem{theorem}{Theorem}
\newtheorem{corollary}[theorem]{Corollary}%Corollaries and Lemmas are numbered as theorems.
\newtheorem{lemma}[theorem]{Lemma}

%These define the format and numbering of definition like environments.
\theoremstyle{definition}%This environment is not in italics, like theorems are.
\newtheorem{definition}{Definition}
\newtheorem*{introduction}{Introduction}%The * means it is unnumbered.
\newtheorem*{conclusion}{Conclusion}

% Put definitions here.  For example, suppose you often use the Greek characters
% alpha, beta, etc., which in LaTeX are \alpha, \beta, etc. (in math mode only).
% Then it may be easier to create shortcuts for these commands, such as:
%\def\aa{\alpha}
%\def\bb{\beta}
% Now, instead of typing \alpha, we can type \aa.
% Here's another I often use:
\def\Bbb#1{{\mathbb #1}} % \Bbb is an obsolete command, but I'm old
% and still use it, so I define it to do what it used to do.  The usage
% is like \Bbb R, which will produce a blackboard bold R, and is literally
% translated to {\mathbb R}.  Note that this command includes a single
% argument.

\begin{document}
\pagenumbering{roman}% Items before Chapter one have roman numbers (if any).
\thispagestyle{empty}%This page has no page number.
\begin{center}
G\"odel's Incompleteness and Gentzen's Inconsistency in Quantum Logic 
\\[48pt]% The title can be no more than 80 characters -- UNLV rule.
 %  The command \\*[48pt] means carriage return with a 48pt gap.  These numbers can be adjusted
 %  to improve the look.

\normalsize by \\*[48pt]

Pushkin Kachroo \\*[48pt]

Doctor of Philosophy \\*[-12pt]%Single space ...
 University of California Berkeley, Virginia Tech\\*[-12pt]
 1993, 2007\\*[48pt]

 A thesis submitted in partial fulfillment\\*[-12pt]
 of the requirements for the \\*[48pt]


 {\bf Master of Science Degree} \\*[-12pt]
 {\bf Department of Physics} \\*[-12pt]
 {\bf College of Sciences} \\*[36pt]

\normalsize
 {\bf Graduate College} \\*[-12pt]
 {\bf University of Nevada, Las Vegas} \\*[-12pt]
 {\bf May 2018}
\end{center} %Replace this file name with the name of your title page.
    % A copyright statement is optional and would be placed here.
    % The copyright page has no page number -- the title page is always page i and the
    % Thesis Approval page is always page ii.
\newpage \setcounter{page}{3} %The Thesis Approval page is page ii.  It is inserted separately.
\chapter*{ABSTRACT}
\addcontentsline{toc}{schapter}{ABSTRACT}% This command adds the Abstract to the table of contents.

\begin{center}
\textbf{G\"odel's Incompleteness and Gentzen's Inconsistency in Quantum Logic}

by

Pushkin Kachroo

 $\langle$Dr. Bernard Zygelman$\rangle$, Examination Committee
 Chair\\*[-12pt]%Single spaced.
 Professor of Physics \\*[-12pt]
 University of Nevada, Las Vegas  %
 \end{center}

This thesis studies the incompleteness theorem as well Gentzen's inconsistency theorems in the context of quantum logic.  As quantum logic is based on a lattice and not Boolean logic, its meta analysis is very different.  This thesis is devoted to studying the differences that this structure provides and how those differences relate to the incompleteness and inconsistency theorems.
 % This is where the abstract goes.

\tableofcontents %inserts table of contents
\listoffigures \addcontentsline{toc}{schapter}{\listfigurename}% Comment these out if there are no figures or tables.
\listoftables \addcontentsline{toc}{schapter}{\listtablename}%

\chapter*{ACKNOWLEDGEMENTS}
\addcontentsline{toc}{schapter}{ACKNOWLEDGEMENTS}

I would like to thank Doina Bein for letting me see what she did
when she wrote her thesis.  I would like to thank Kensaku Umeda,
who helped me with the UNLV requirements.
  %Acknowledgements come after the tables.

\newpage % Do not remove this command.  It's there to make sure the page numbering is correct.
\pagenumbering{arabic} %Chapter 1 begins on page 1.

\chapter{INTRODUCTION}% Chapter titles are capitalized -- UNLV rule.

Writing a graduate thesis is very time consuming.  One of the
obstacles placed in front of the student is the set of
requirements for the look of thesis.  These requirements look like
they date from the middle ages, but there's nothing you or I can
do about that.  Writing a thesis in mathematics or some other
discipline that requires mathematical symbols is even more
difficult, because most typesetting programs do not deal with them
very well.  One program that was designed to handle mathematical
symbols is \LaTeX.  This program is a publishing grade program.
For example, I have published a book that was typeset using \LaTeX
\cite{B1} and almost all journals that I submit to request that
source files be in \LaTeX.

Unfortunately, \LaTeX\ is much more sophisticated than what the
average graduate student requires, and none of its standard
packages look anything like the required UNLV format (though every
one of them looks better). On the other hand, much of this can be
automated by a \TeX\! pert. That's what this package is about.  It
describes how to use the UNLVthesis.sty style file to format
theses into the UNLV format. Unfortunately, I do not make the
rules, so there may be things I haven't anticipated or dealt with
correctly, so some tweaking may be required.  I welcome questions,
though I may not deal with them promptly.  I can be reached at
baragar@unlv.edu.

There are some requirements that are very difficult to adhere to.
For example, the font style is to be ``a standard ... computer
font, such as Bookman, Courier, Arial(Helvetica), or Times.''
Though it is possible to change the font in \LaTeX\ documents, I
prefer to not do so.  The \LaTeX\ font is Computer Modern Roman,
and very few would notice that it is not Times.
 %Replace these with your chapters.
%%%%%%%%%%%%%%%%%%%%%%%
\bibliographystyle{plain}
\bibliography{PKPhysMS}
%%%%%%%%%%%%%%%%%%%%%%%
\begin{spacing}{1} %This makes this page single spaced.
\thispagestyle{plain} % This puts the page number at the bottom.
\begin{center}
 VITA\addcontentsline{toc}{schapter}{VITA}\\*[2\baselineskip]% Don't use \chapter* -- the Vita begins
    % at the top of the page.
 Graduate College \\
 University of Nevada, Las Vegas \\*[\baselineskip]
 Arthur Baragar\\*[\baselineskip]
\end{center}

%\noindent Local Address:\\ % Include this if different from your home address.
%***\\*[2\baselineskip]

\noindent Home Address:

 8815 Wallaby Lane

 Las Vegas, Nevada 89123\\*[.5\baselineskip]

\noindent Degrees:

 Bachelor of Science, Honors Mathematics, 1985

 University of Alberta, Edmonton, Canada\\*[.5\baselineskip]

 Doctor of Philosophy, Mathematics, 1991

 Brown University, Providence, Rhode Island\\*[.5\baselineskip]

%\noindent Special Honors and Awards: % Omit, if none.

%\noindent Publications:  % Omit, if none.

\noindent  Dissertation Title: %(Or Thesis title)
The Markoff Equation and Equations of Hurwitz\\*[.5\baselineskip]

\noindent  Dissertation Examination Committee: %(Or Thesis ...)

 Chairperson, Dr.~Joseph H. Silverman, Ph.D. %Put ~ after a period if the period should not
    % be interpreted as an end of sentence.  This will restrict how much space can be there,
    % and will prevent a line break there.  This is not necessary if there is no space after
    % the period, for example, in www.unlv.edu

 Committee Member, % I don't remember anymore ...

 Committee Member,

 Graduate Faculty Representative,

\end{spacing}
  %The Vita is the last page.

\end{document}
