\chapter{WHERE DO I GET \LaTeX?}

\LaTeX\ is a freeware application.  It was originally designed by
Donald Knuth at Stanford, and is being maintained by the (mostly
European) academic community.  There are versions that run on PC,
Mac, or unix systems.  Though the compilers are platform
dependent, the source files are not.

\section{Obtaining a PC version of \LaTeX}

The PC version of \LaTeX\ is MikTeX.  It can be obtained from
\[
\hbox{www.miktex.org}
\]
As mentioned, it is free.  I strongly recommend the use of the
`front end' editor WinEdt, which is shareware and can be obtained
from
\[
\hbox{www.winedt.com}
\]
The cost is nominal (currently \$40).  Their website also has
links to the necessary downloads.  Besides MikTeX and WinEdt, one
should get Ghostscript and Ghostview (go to the WinEdt website for
links).  These too are free.

\section{Macintosh versions}

Sorry, you're on your own.

\section{Unix versions}

Again, you're on your own.  However, chances are you're using unix
on a mainframe, in which case someone else has probably already
installed \LaTeX.

\section{Getting started}

I suggest using the electronic versions of these files as a
template.  Begin by editing it.  But keep a copy of these, since
in the next chapter, I do several things that are probably
representative of the sort of things you probably will want to do.
You can look at the .tex file to see how I did them.

\section{Resources}

There are several books that I recommend.  I like {\it The \LaTeX\
Companion} by Gossens, Mittelbach, and Samarin \cite{G}, but this
is probably more for an expert. I used to prefer {\it A document
preparation system \LaTeX} by Leslie Lamport \cite{L}, which means
it is more suited to the beginner. I've also heard good things
about {\it A guide to \LaTeX2e, Document Preparation for Beginners
and Advanced Users}, by Helmut Kopka and Patrick Daly \cite{K-D},
but I think this too is for the expert. Finally, one can look at
the starter page for the website
\[
\hbox{www.ctan.org}
\]
This includes links to {\it The (Not so) short introduction to
\LaTeX2$\epsilon$}, by Tobias Oiteker, et al, a 110 page free
document \cite{O}. I haven't tried using this document, but it
looks pretty good.
