\chapter[A sample document, eh?]{A SAMPLE DOCUMENT}%Titles must be capitalized.

Before we begin, take a quick look at how this chapter appears in
the table of contents.  This is because of the `optional' argument
in the chapter entry.  Also, while reading this chapter, please
also read its source file SampleChapter3.tex.  Many of the
comments refer to what is written in the source file.

A double return is interpreted as a new paragraph. A single return
is ignored (look at the source file).

To type an equation in the line, put it between dollar signs.  For
example, $a^2+b^2=c^2$.  If it's very important, display the
equation by using backslash square braces:
\[
E=mc^2.
\]
If the equation is to be referenced, then do the following:
\begin{equation}\label{Equation 1}%This is the label of the equation -- LaTeX will number it.
a^n+b^n=c^n.
\end{equation}
(That instruction doesn't make much sense unless you are also
looking at the .tex file, which in this case, is
SampleChapter3.tex.)  Equation (\ref{Equation 1}) has no
non-trivial solutions in the integers. Some results are important:

\begin{theorem}[Fermat's Last Theorem]\label{theorem 1}  The equation
\[
x^n+y^n=z^n
\]
has no solutions $x,y,z\in \Bbb Z$ for $n\geq 3$ and $xyz\neq 0$.
\end{theorem}

Theorem \ref{theorem 1} was proved by Wiles.  The proof does not
appear in \cite{B1}.

Sometimes you might want to have aligned equations, like the
following:
\begin{align*}%the * means this equation doesn't have a label.
\hbox{Force}&=(\hbox{mass})(\hbox{acceleration}) \\ %the \\ tells it to insert a line break.
F&=ma. % the & is the alignment character.
\end{align*}
The * in \verb \begin{align*} \ in the .tex file means that these
equations have no numbers.  If they are to be numbered, leave the
* out:
\begin{align}
\hbox{Force}&=(\hbox{mass})(\hbox{acceleration}) \\
F&=ma.
\end{align}
If you want only one of these equations numbered, try the
following:
\begin{align}
\hbox{Force}&=(\hbox{mass})(\hbox{acceleration}) \notag\\ %the \notag means no tag here.
F&=ma.
\end{align}

A picture is shown in Figure \ref{fig1}.
\begin{figure}[h]%This [h] tells LaTeX to try to put the picture here.
                 %Without it, it will go to the top of the next page.
\begin{center}
{\mbox{\includegraphics[height=140pt]{fig1.eps}}}
\end{center}
\caption{\label{fig1}Isn't this a nice picture?}
\end{figure}% Be careful with empty lines -- remember, they mean new paragraph.
    % There are no empty lines before and after the figure environment since
    % I do not want a new paragraph here.
For more information about graphics, consult documentation on
graphicx (run a search on the www.ctan.org site).  Support of
graphics is really a function of the dvi viewer.  As far as I
know, the only type of graphic supported by the standard viewer is
the .eps format.  Adobe Illustrator\footnote{Adobe Illustrator is
a rather expensive product.  This is the way one does footnotes.
More text here. } can convert most graphics files to .eps files.
Figure \ref{fig1}, though, was produced using Maple and saved as a
\LaTeX\ file (which includes a .eps component).

Though I always use Illustrator\footnote{Another footnote.} to
create pictures, I understand that any printable graphic file can
be converted to .eps format. The idea is this -- if it is
printable, then it can be printed on a postscript printer.  Any
computer can be set up to print to a postscript printer, even if
not connected to one.  So, obtain a generic postscript printer
driver (free from Adobe at www.adobe.com) and set it up to print
to a file.  While setting it up, choose the option to send files
to the ``printer'' in .eps format (the default is .ps format).  I
think I saw full instructions for this in \cite{O}.

For those interested, the commands in Maple that produces
\ref{fig1} are:
\begin{trivlist}
\item plot(\{[2*cos(t),sin(t), t=0..2*Pi],
[-1+cos(t),.2+sin(t),t=0..2*Pi]\}, axes=none, scaling=constrained,
color=black);
\end{trivlist}
